\section{Sentence, Proposition and Expression}

\begin{definition}[Sentence]
    A sentence is: \textcolor{red}{In first place, put it all definitions about sentence that I could think or find in other references!}

    \begin{itemize}
        \item Expression of a complete thought. \ref{exemplo1}
    \end{itemize}

\end{definition}

The sentences can be the following:

\begin{itemize}
    \item Affirmative
    \item Negative
    \item Imperative
    \item Exclamation
    \item Interrogative
\end{itemize}

\begin{definition}[Open Sentence]
A sentence is ...


\end{definition}

\begin{definition}[Close Sentence]
A sentence is ...


\end{definition}

\begin{remark}[Expression]
    The phrase that is not a sentence.
\end{remark}

\begin{definition}[Predicate]
    A predicate is...
\end{definition}

\begin{definition}[Proposition]
A proposition is a statement (communication) that is either true or false.
\end{definition}

Is not considered a proposition the following sentences:

\begin{itemize}
    \item Interrogative
    \item Exclamation
    \item Open sentences
    \item Imperative sentences
    \item Without verb
\end{itemize}

So, what is the key to understand about proposition?? I gave above its definition, but I don't show you yet the mechanism that makes a given sentence to acquire its dichotomous nature.

Where we could to pay our attention to understand if a given sentence has or not dichotomous nature?

Well, in first place, we have to check if a given sentence is or not objective. That is, this sentence doesn't have any subjective feature. For example, if I give the following affirmation:

\begin{example}\label{exemplo1}
    You are beautiful.
\end{example}

The example above is a sentence, but is not a proposition. Why?

Because, the subject "You" is not defined.

Ok! So, we have to give an subject to this "You". So, if we affirm the following:

\begin{example}
    Wonder Woman is beautiful.
\end{example}

Now, the sentence above became a proposition? No! So why?

Because there is a person who think she is beautiful and there is ones that doesn't agree!

Ok, ok! I got it. So, if everyone agrees the sentences become a proposition? For example, as following sentence:

\begin{example}[\textcolor{red}{A sentence that is subjective and everyone agree (this is very difficult)}]
    \textcolor{red}{Thinking}
\end{example}

Finally, the sentence above is an proposition, right? NOOOOO!!! Is not a proposition, because same any person thinks that the affirmation above is right, we can't see, measure or observe to conclude universally that is true.

Well, an objective sentence is possible to verify, measure or observe? Well, every sentence where it has these feature is objective and is a proposition, but this still doesn't define a proposition.

So, why the sentences where it has features above is a proposition?

Because, the features that is possible to verify, measure or observe its conclusion is the same for any context. For example:

\begin{example}[Feature that is possible to verify]
    \textcolor{red}{Thinking}
\end{example}

\begin{example}[Feature that is possible to measure]
    \textcolor{red}{Thinking}
\end{example}

\begin{example}[Feature that is possible to observe]
    \textcolor{red}{Thinking}
\end{example}

All those examples above is a proposition. But, this is not enough condition to be equivalent to the definition given about proposition. Because, there is an sentence that is objective and affirmative, but is not possible to verify, measure or observe, but is a proposition. Be careful it isn't going to confound with Axioms/Postulates, these have features mentioned above less to be true or false.

\begin{example}[A sentence that is proposition, but is not possible to verify, measure or observe.]
    \textcolor{red}{Thinking}
\end{example}

\begin{example}[Conjecture - An proposition that still not proof if is true or false]
    \textcolor{red}{Thinking}
\end{example}

\begin{definition}[Mathematical proof]
A mathematical proof of a proposition is a chain of logical deductions
leading to the proposition from a base set of axioms.

\end{definition}

\subsection{Binary Connectives}

\subsection{Unary Connectives}

% Use apenas o arquivo de referências específico desta seção
\printbibliography[heading=subbibliography, title={Referências da Seção 1.1}, keyword=chapter1secao1]